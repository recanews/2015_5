\documentclass{book}

%*****************


\usepackage[spanish]{babel}
\usepackage[utf8]{inputenc}

\usepackage{graphicx}
\usepackage{lipsum}
\usepackage{microtype}
\usepackage{enumerate}

\usepackage[T1]{fontenc}
\usepackage{lmodern}

\usepackage[pdftex]{hyperref}

\hypersetup{pdfauthor={J. S. Castellanos-Durán},pdftitle={RECANEWS Volumen 3 - Marzo 2015},colorlinks,linkcolor=black,urlcolor=blue}

\usepackage[paperwidth=210mm, paperheight=297mm, textwidth=160mm, textheight=240mm, bindingoffset=1cm]{geometry}


% obliczenie szerokości lewego marginesu
\usepackage{calc}
\newlength{\lmargin}
\setlength{\lmargin}{1in + \hoffset + \oddsidemargin}

\usepackage{flowfram}

\usepackage{color}

\usepackage{tikz}
\usepackage{anyfontsize}

% definicja ramek typu flow umieszczonych na stonie 1
\newflowframe[1]{8cm}{24\baselineskip}{-.50cm}{0\baselineskip}[frame1-1a]
\newflowframe[1]{8cm}{23\baselineskip}{8.5cm}{0\baselineskip}[frame1-2b]
%\newflowframe[1]{5cm}{27\baselineskip}{11cm}{0\baselineskip}[frame1-3c]

%definicja ramek statycznych wstawianych na stronie 1
\newstaticframe[1]{\paperwidth}{14cm}{-\lmargin}{12.5cm}[frameS-1a]
\newstaticframe[1]{14cm}{7\baselineskip}{0cm}{45\baselineskip}[frameS-1b]

%definicja ramki dymamicznej wstawiania na stonie nieparzystej
\newdynamicframe[odd]{2cm}{2cm}{-\lmargin}{6cm}[frameD-1a]
%definicja ramki dymamicznej wstawiania na stonie parzystej
\newdynamicframe[even]{2cm}{2cm}{\textwidth+\lmargin-2cm}{6cm}[frameD-1b]

% definicja ramek typu flow na kolejnych stronach
\newflowframe[>1]{8cm}{57\baselineskip}{-.50cm}{0\baselineskip}[frame2-1a]
\newflowframe[>1]{8cm}{57\baselineskip}{8.5cm}{0\baselineskip}[frame2-2a]
%\newflowframe[>1]{5cm}{57\baselineskip}{11cm}{0\baselineskip}[frame2-3a]

\definecolor{green}{rgb}{0.6,0.8,0.1}
\definecolor{cafe}{rgb}{0.8,0.2,0.2}

\title{RECANEWS Volumen 5 - Mayo 2015}
\author{J. Sebastián Castellanos Durán}
\date{\relax}

 
\begin{document}
\newcommand{\theimage}{fig22.jpg}
\input{head} 
\tableofcontents{}
%*****************

%*********************************************
\addcontentsline{toc}{section}{Noticias Astronomía y ciencias del espacio en Colombia}
\section*{Noticias Astronomía y ciencias del espacio en Colombia}
%*********************************************
\subsection{2 posiciones postdoc en la UIS}
La Universidad Industrial de Santander está buscando dos postdocs en

Astrofísica relativista \url{https://inspirehep.net/record/1352755}
Astropartículas \url{https://inspirehep.net/record/1348393}

Cierre de la convocatoria 31 de Mayo. 
Condiciones: \url{http://goo.gl/28vpfp}

Contacto: Luis A. Núñez 
Correo: \url{lnunez@uis.edu.co}


%---------------------------------
\subsection{Segundo Workshop Astronomía en los Andes - Bogotá, Colombia}


A finales de Julio tendrá lugar en Bogotá el Segundo Workshop de Astronomía en los Andes, un evento del Nodo Andino (OAD/IAU) en el cual astrónomos de la región andina nos reuniremos para organizar, potenciar colaboraciones y conocer capacidades. Adicionalmente, se firmarán los convenios donde se oficializa la creación de la Oficina Regional Andina de Astronomía para el Desarrollo de la Unión Astronómica Internacional, pues contamos con la presencia de Kevin Govender, director de la OAD-IAU. \\
\\
Entre los temas a tratar durante el evento están Astropartículas, Radioastronomía, Educación a nivel de pre- y pos-grado, Educación a nivel básico y Divulgación. También se aprovechará la ocasión para tratar temas locales como la organización de la LARIM Cartagena 2016, CAP2016 y otras discusiones importantes. La información completa sobre el LOC/SOC del evento, un cronograma provisional, invitados, etc. se encuentra en la página:
\begin{center}
\url{http://goo.gl/aT9VIX}
\end{center}
Están todos invitados a registrarse con un Resumen antes del 15 de Mayo en el siguiente enlace:
\begin{center}
\url{http://goo.gl/XpsH1r}
\end{center}
La inscripción no tiene costo.

\begin{flushright}
Germán Chaparro 

\url{gchaparrom@ecci.edu.co}

Jaime Forero 

\url{je.forero@uniandes.edu.co}
\end{flushright}


%---------------------------------
\subsection{Nueva página de astronomía en Colombia}
Esta pagina busca resumir las principales actividades de astronomía en Colombia. 
 
 \begin{description}
 \item[Github: ]ColombianAstronomy
 \end{description}
\begin{center}
\url{http://goo.gl/VUMPxH}
\end{center}
\begin{flushright}
Nicolas Garavito

U. Andes
\end{flushright}
%---------------------------------

\subsection{Astronomía al Aire}
Astronomía al Aire es un programa radial del grupo de astronomía de la Universidad Industrial de Santander. Durante este mes cumplieron sus primeros 6 programas: 
\begin{itemize}
\item Agujeros Negros: el triunfo de la gravedad.
\item Relatividad y GPS.
\item El universo y la ciencia.
\item El elusivo neutrino.
\item La gravedad y la geometría.
\item La noche oscura y la paradoja de Olbers.
\end{itemize}
Los programas se pueden encontrar en
\begin{center}
\url{halley.uis.edu.co/aire/}
\end{center}

%*********************************************
\addcontentsline{toc}{section}{Escuelas}
       \section*{Escuelas}
%*********************************************




%*********************************************
\addcontentsline{toc}{section}{Programas de posgrados}
       \section*{Programas de posgrados}
%*********************************************

\subsection{Maestría en Balseiro}
Están abiertas las inscripciones para el programa de Diploma/Maestría en el Instituto Balseiro para el período que comienza el 27 de julio de de 2015. Detalles del programa pueden encontrarse en la página

\url{http://fisica.cab.cnea.gov.ar/particulas/diploma}


%*********************************************
\addcontentsline{toc}{section}{Congresos y eventos}
\section*{Congresos y eventos}
%*********************************************



%*********************************************
\addcontentsline{toc}{section}{Opinión}
\section*{Opinión}
%*********************************************





%*****************
\input{notes}
\end{document}
%*****************


