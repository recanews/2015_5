\documentclass{book}

%*****************


\usepackage[spanish]{babel}
\usepackage[utf8]{inputenc}

\usepackage{graphicx}
\usepackage{lipsum}
\usepackage{microtype}
\usepackage{enumerate}

\usepackage[T1]{fontenc}
\usepackage{lmodern}

\usepackage[pdftex]{hyperref}

\hypersetup{pdfauthor={J. S. Castellanos-Durán},pdftitle={RECANEWS Volumen 3 - Marzo 2015},colorlinks,linkcolor=black,urlcolor=blue}

\usepackage[paperwidth=210mm, paperheight=297mm, textwidth=160mm, textheight=240mm, bindingoffset=1cm]{geometry}


% obliczenie szerokości lewego marginesu
\usepackage{calc}
\newlength{\lmargin}
\setlength{\lmargin}{1in + \hoffset + \oddsidemargin}

\usepackage{flowfram}

\usepackage{color}

\usepackage{tikz}
\usepackage{anyfontsize}

% definicja ramek typu flow umieszczonych na stonie 1
\newflowframe[1]{8cm}{24\baselineskip}{-.50cm}{0\baselineskip}[frame1-1a]
\newflowframe[1]{8cm}{23\baselineskip}{8.5cm}{0\baselineskip}[frame1-2b]
%\newflowframe[1]{5cm}{27\baselineskip}{11cm}{0\baselineskip}[frame1-3c]

%definicja ramek statycznych wstawianych na stronie 1
\newstaticframe[1]{\paperwidth}{14cm}{-\lmargin}{12.5cm}[frameS-1a]
\newstaticframe[1]{14cm}{7\baselineskip}{0cm}{45\baselineskip}[frameS-1b]

%definicja ramki dymamicznej wstawiania na stonie nieparzystej
\newdynamicframe[odd]{2cm}{2cm}{-\lmargin}{6cm}[frameD-1a]
%definicja ramki dymamicznej wstawiania na stonie parzystej
\newdynamicframe[even]{2cm}{2cm}{\textwidth+\lmargin-2cm}{6cm}[frameD-1b]

% definicja ramek typu flow na kolejnych stronach
\newflowframe[>1]{8cm}{57\baselineskip}{-.50cm}{0\baselineskip}[frame2-1a]
\newflowframe[>1]{8cm}{57\baselineskip}{8.5cm}{0\baselineskip}[frame2-2a]
%\newflowframe[>1]{5cm}{57\baselineskip}{11cm}{0\baselineskip}[frame2-3a]

\definecolor{green}{rgb}{0.6,0.8,0.1}
\definecolor{cafe}{rgb}{0.8,0.2,0.2}

\title{RECANEWS Volumen 5 - Mayo 2015}
\author{J. Sebastián Castellanos Durán}
\date{\relax}

 
\begin{document}
\newcommand{\theimage}{fig22.jpg}
\input{head} 
\tableofcontents{}
%*****************

%*********************************************
\addcontentsline{toc}{section}{Noticias Astronomía y ciencias del espacio en Colombia}
\section*{Noticias Astronomía y ciencias del espacio en Colombia}
%*********************************************
\subsection{LARIM - Inscripción}

Ya esta implementada la plataforma de registro como participante de nuestra LARIM 2016
\begin{center}
\url{http://larim.unal.edu.co}
\end{center}
\textbf{¡Es de suprema importancia motivar a nuestros colegas y estudiantes a que hagan su pago durante la inscripción temprana!}
\begin{itemize}
\item Consignación
\end{itemize}
Consignar a nombre del titular Academia Colombiana de Ciencias Exactas, Físicas y Naturales\\
Nit: 860.026.6351\\
Cuenta de Ahorros: 055000 99 00 29 2575\\
Banco: DAVIVIENDA\\
País: Colombia\\
Una vez realizada su consignación, ingrese a la plataforma de registro, diligencie las casillas y  suba un archivo adjunto que contenga la imagen, claramente legible, del recibo de la consignación. 
\begin{itemize}
\item Costos
\end{itemize}
\textbf{Antes del 1 de Mayo de 2015}\\
Estudiantes (carnet vigente Octubre-2016): 250.000 pesos*\\
Profesionales 400.000 pesos*\\
*Incluye sin costo adicional la participación en la cena del evento.
\newpage
\subsection{Identification of Potential Sites for Astronomical Observations in Northern South-America}

Después de más de cuatro años de riguroso trabajo, el profesor Giovanni Pinzon del Observatorio Astronómico Nacional junto al Magister Danilo González concluyeron la primera etapa de estudio de los posibles lugares para ubicar un telescopio profesional en Colombia. Los resultados de este importante trabajo se pueden consular en
\begin{center}
\url{http://arxiv.org/abs/1504.06033}
\end{center}


\subsection{2 posiciones postdoc en la UIS}
La Universidad Industrial de Santander está buscando dos postdocs en\\
\\
Astrofísica relativista\\ \url{https://inspirehep.net/record/1352755}\\
Astropartículas\\ \url{https://inspirehep.net/record/1348393}
\\
\\
Cierre de la convocatoria 31 de Mayo. \\
Condiciones: \url{http://goo.gl/28vpfp}
\\
Contacto: Luis A. Núñez \\
Correo: \url{lnunez@uis.edu.co}

\subsection{Astronomía al Aire}
Astronomía al Aire es un programa radial del grupo de astronomía de la Universidad Industrial de Santander. Durante este mes cumplieron sus primeros 6 programas: 
\begin{itemize}
\item Agujeros Negros: el triunfo de la gravedad.
\item Relatividad y GPS.
\item El universo y la ciencia.
\item El elusivo neutrino.
\item La gravedad y la geometría.
\item La noche oscura y la paradoja de Olbers.
\end{itemize}
Los programas se pueden encontrar en
\begin{center}
\url{halley.uis.edu.co/aire/}
\end{center}
%---------------------------------
\newpage
\subsection{Segundo Workshop Astronomía en los Andes - Bogotá, Colombia}


A finales de Julio tendrá lugar en Bogotá el Segundo Workshop de Astronomía en los Andes, un evento del Nodo Andino (OAD/IAU) en el cual astrónomos de la región andina nos reuniremos para organizar, potenciar colaboraciones y conocer capacidades. Adicionalmente, se firmarán los convenios donde se oficializa la creación de la Oficina Regional Andina de Astronomía para el Desarrollo de la Unión Astronómica Internacional, pues contamos con la presencia de Kevin Govender, director de la OAD-IAU. \\
\\
Entre los temas a tratar durante el evento están Astropartículas, Radioastronomía, Educación a nivel de pre- y pos-grado, Educación a nivel básico y Divulgación. También se aprovechará la ocasión para tratar temas locales como la organización de la LARIM Cartagena 2016, CAP2016 y otras discusiones importantes. La información completa sobre el LOC/SOC del evento, un cronograma provisional, invitados, etc. se encuentra en la página:
\begin{center}
\url{http://goo.gl/aT9VIX}
\end{center}
Están todos invitados a registrarse con un Resumen antes del 15 de Mayo en el siguiente enlace:
\begin{center}
\url{http://goo.gl/XpsH1r}
\end{center}
La inscripción no tiene costo.

\begin{flushright}
Germán Chaparro 

\url{gchaparrom@ecci.edu.co}

Jaime Forero 

\url{je.forero@uniandes.edu.co}
\end{flushright}
%---------------------------------
\subsection{Nueva página de astronomía en Colombia}
Esta pagina busca resumir las principales actividades de astronomía en Colombia. 
 
 \begin{description}
 \item[Github: ]ColombianAstronomy
 \end{description}
\begin{center}
\url{http://goo.gl/VUMPxH}
\end{center}
\begin{flushright}
Nicolas Garavito

U. Andes
\end{flushright}
%---------------------------------
\newpage
\subsection{1st workshop on Current Challenges in Cosmology: Inflation and the Origin of CMB anomalies.\\ Mayo 18 - 22, 2015 Cali, Colombia}
The Workshop is intended to gather leading experts in theoretical Cosmology to share knowledge, discuss new theoretical developments in inflationary cosmology and ultimately elucidate the puzzle of CMB anomalies. 
\textbf{Main topics}
\begin{itemize}
\item Statistical anisotropy and anisotropic expansion
\item Primordial magnetic fields
\item Parity violation and polarization in the CMB
\item Effective field theory of inflation
\item Anisotropic non-gaussianity
\item CMB data analysis
\item Primordial Gravitational  Waves
\item Galileons and Horndeski theories
\end{itemize}
\par
\noindent Invited Speakers:\\
\\
Konstantinos Dimopoulos (Lancaster University),	David F. Mota (University of Oslo), Maresuke Shiriashi (Università degli Studi di Padova), Patrick Peter
(Institut d’Astrophysique de Paris), Azadeh Maleknejad (Institute for Research in Fundamental Sciences), Marco Peloso (University of Minnesota), Lorenzo Sorbo (University of Massachusetts), Thiago S. Pereira (Universidade Estadual de Londrina), Leonardo Senatore* (Stanford University), Mindaugas Karciauskas (Helsinki University), Martin Bucher (University Paris 7 Diderot).\\
\\
Más información

\begin{center}
\url{cosmology.univalle.edu.co/index.php/home}
\end{center}



%---------------------------------

\newpage

\subsection{Astropuerta - Mayo}

\textbf{Principales eventos celestes de Mayo 2015}

\begin{itemize}
\item     Lunes 4 - Luna llena
\item     Martes 5 - Conjunción de la Luna y Saturno
 \item    Jueves 7 - Elongación máxima Este de Mercurio
\item     Lunes 11 - Luna en cuarto menguante
\item     Viernes 15 - Ocultación de Urano por la Luna visible en el centro de Sudamérica y en el centro de África
\item     Lunes 18 - Luna nueva
\item     Sábado 23 - Oposición de Saturno
\item     Lunes 25 - Luna en cuarto creciente
\end{itemize}

\textbf{Principales efemérides históricas de Mayo 2015} 

\begin{itemize}
\item     Viernes 1 - 1949: Gerard Kuiper descubre a Nereida, luna de Neptuno
 \item    Martes 5 - 1961: Alan Shepard, primer estadounidense en el espacio exterior
 \item    Jueves 7 - 1925: Operación del primer proyector de planetario en Munich, Alemania.
 \item    Jueves 14 - 1973: Lanzamiento de la estación espacial Skylab
 \item    Sábado 30 - 1975: Fundación de la Agencia Espacial Europea
 \end{itemize}   
 
\subsection{Cursos de Herramientas y Métodos Computacionales.}


La Universidad de los Andes, Dicta cada semestre los cursos de Herramientas
Computacionales y Métodos computacionales. El primero se dicta para estudiantes
de primero a tercer Semestre de Física y Geociencias. El objetivo del curso es desarrollar habilidades basicas de programación en un lenguaje de alto nivel como Python, se eneseñan algunos métodos de análisis númerico y exibe algunas herramientas utíles de análisis de datos.\\
\\
Métodos computacionales se dicta para estudiantes de cuarto a septimo semestre. El objetivo de este curso es profundizar en los métodos computacionales para que los estudiantes esten en capacidad de abordar cualquier problema. En el curso se profundiza en el uso C, Python  y R.\\
\\
En el siguiente link encontrarán todo el material (Syllabus, Clases y Tareas)
de los cursos.\\
\\
\url{http://computationalscienceuniandes.github.io/}
\newpage 
 
 
\subsection{Charlas Spacio}
\textbf{Mayo 2}
\begin{itemize}
    \item 4:00pm Cómo obtener y manipular datos astronómicos:
\begin{itemize}
    \item Jaime Forero: Como obtener datos de la simulacion Cosmologica Illustris y del Sloan Digital Sky Survey.
    \item Sebastian Castellanos: Datos del Sol,  Observatorio Astronómico UNAL.
    \item Oscar Restrepo. Uso de la base de datos NED (https://ned.ipac.caltech.edu). Universidad ECCI
    \item Fabian Saavedra: Uso de datos de sensoramiento remoto planetario. Universidad Nacional de Colombia.
    \end{itemize}
\end{itemize}


\textbf{Mayo 9}
\begin{itemize}
   \item 4:00pm Fulguraciones solares y el campo magnético, Juan Sebastian Castellanos, Observatorio Astronómico UNAL.
   \item 5:15pm Cráteres de impacto: qué son y cómo identificarlos, Maria Caferino, UNAL
\end{itemize}
\textbf{Mayo 23}
\begin{itemize}
   \item 4:00pm Una noche en el observatorio, Karla Peña, Pontificia Universidad Católica de Chile (Skype)
   \item 5:15pm Actividad Solar-Clima Terrestre-Cerros Orientales de Bogotá.
    , Juan Manuel Alvarez, WF\&E Foundation
    \end{itemize}


%*********************************************
\addcontentsline{toc}{section}{Escuelas}
       \section*{Escuelas}
%*********************************************

\subsection{Summer Visiting Program at CfA}

 \textbf{Summer Visiting Program for Astronomers Harvard-Smithsonian Center for Astrophysics (CfA)}\\
\\
\textbf{Overview}\\
In this program, astronomers from middle- and low-income countries can apply to spend a month in the northern summer at the Harvard-Smithsonian Center for Astrophysics (CfA). Travel and basic living expenses will be covered by the grant.\\
\\
In their application, candidates have to present a work plan in line with the projects proposed by the mentors (see below) and outline the benefits of their visit for the continuation of their career and the development of astronomy back in their home country. Two visiting scientists will be selected for the summer of 2015. Two local mentors have proposed a scientific project (see below).\\
\\
We foresee mutual benefit from such a visiting program for both the visiting scientist and the CfA scientists and staff, who will have the opportunity to learn from the visitors about the status of astronomy in their countries. From such visits lasting collaborations can grow that foster knowledge exchange and education (e.g., workshops/schools later organized in the country of the visitor), and increase integration, all high priorities for the International Astronomical Union and its Office of Astronomy for Development.
\begin{center}
\url{http://goo.gl/GRLd0O}
\end{center}
%*********************************************
\addcontentsline{toc}{section}{Programas de posgrados}
       \section*{Programas de posgrados}
%*********************************************
\subsection{39 PhD positions in Stockholm }
\begin{center}
\url{http://goo.gl/sOrlGO}
\end{center}
\subsection{Maestría en Balseiro}
Están abiertas las inscripciones para el programa de Diploma/Maestría en el Instituto Balseiro para el período que comienza el 27 de julio de de 2015. Detalles del programa pueden encontrarse en la página
\begin{center}
\url{http://goo.gl/mhuw8o}
\end{center}
%.........................
\subsection{PhD positions at University of Roma Tor vergata}
 PhD general call for 14 PhD positions (12 out of 14 with scholarship) at the University of Roma Tor vergata, in convention with the University of Roma Sapienza and INAF. 

The poster is available at 
\begin{center}
\url{https://goo.gl/ZgM3ee}
\end{center}

Positions are available in the following research areas: 
 \begin{enumerate}[i)]
 \item Galactic and extragalactic Astrophysics
 \item Gravitation and Cosmology
 \item Solar and Stellar Physics
 \end{enumerate}
Application forms and instructions can be found at
\begin{center}
\url{http://goo.gl/UIFBdm}
\end{center}

%--------------
\subsection{PhD Offer in Astrochemistry}
Tipo/Type: Predoctoral
Centro/Institution: Observatoire de Paris and Université de Cergy Pontoise
País/Country: France
Fecha limite/Deadline: 15/06/2015
Descripción/Description: \\
 \\
The Observatoire de Paris and the Université de Cergy Pontoise offer one PhD position in the field of astrochemistry.\\
 \\
We are looking for candidate interested in laboratory experiments and studies of the interstellar medium. A solid background in physics and some interest in chemistry is required.\\
 \\
The succesful candidate will work on a project aimed at understanding the origin of the differential depletion of N2 and CO in prestellar cores.\\
 \\
Part of the work will be done under the supervision of Laurent Pagani, Directeur de Recherche at LERMA/Obs Paris, and will consist of analysis of observational data from dense cores and modeling of the chemistry.\\
 \\
A second part of the work will consist in laboratory astrochemistry and will be carried out at Cergy-Pontoise under the supervision of François Dulieu.
\begin{center}
\url{http://goo.gl/0j0U9R}
\end{center}

%---------------
\subsection{PhD and Masters projects at MPI}

Several PhD and Masters student projects are available in the area of gamma-ray astronomy at the MPIK. The divisions of Professors Jim Hinton and Werner Hofmann are working together on the Cherenkov Telescope Array (CTA) and High Energy Stereoscopic System (H.E.S.S.) very high energy gamma-ray telescope systems. Opportunities include development and evaluation of camera hardware, data analysis, and simulations for CTA, and data analysis and scientific exploitation of H.E.S.S. phase-2, including associated multi-wavelength and multi-messenger activities. 
\begin{center}
\url{https://goo.gl/Lr0GK1}
\end{center}
\newpage

%*********************************************
\addcontentsline{toc}{section}{Opinión}
\section*{Opinión}
%*********************************************
\subsection{Llegó el 'verano'}
RECANEWS en este volumen quiere felicitar a los estudiantes colombianos que empiezan su doctorado o han sido escogidos para realizar estancias de investigación.

\begin{description}
\item[Sebastián Bustamante - U. Antioquia]. \\
Doctorado en la Universidad de Heidelberg y el instituto Max Planck gracias a una beca del DAAD 
\item[Maria Camila Remolina - U. Andes].\\
Estancia de investigación LEAPS en Leiden, Paises Bajos.
\item[Saida  Díaz- Observatorio Astronómico Nacional].\\
Estancia de investigación Summer Internship in Astrophysics and Astronomical Instrumentation at INAOE Puebla, Mexico.
\item[Jesus David Prada - U. Andes].\\
Estancia de investigación en la Universidad de Cornell gracias a un convenio con la U. Andes.
\item[Sebastián Castellanos-Durán - ].\\
\textbf{Observatorio Astronómico Nacional}.\\
Estancia de investigación IAESTE, Suiza.
\item[Felipe Gomez - U. Andes].\\
Pasantia de Maestría por convenio con Colciencias- U. Purdue
\end{description}

 En nombre de RECA les deseamos los mejores éxitos y felices viajes. Los estudiantes son
 \subsection{De estudiantes para estudiantes}
 Resumen de artículos científicos escritos por estudiantes para estudiantes
y consejos académicos
\begin{center}
\url{http://astrobites.com/}
\end{center}
\subsection{Tips}
Tips para astrónomos profesionales
\begin{center}
\url{http://www.astrobetter.com/ }
\end{center}
%*****************
\input{notes}
\end{document}
%*****************


